\documentclass[10pt,a4paper]{article}
\usepackage[T1]{fontenc}
\usepackage[utf8]{inputenc}
\usepackage{graphicx}
\usepackage{tabularx}
\usepackage{helvet}
\usepackage{bera}
\usepackage{hyperref}
\usepackage[a4paper,margin=1in]{geometry}
\usepackage[polish]{babel}
\usepackage{float}
\usepackage{listings}
\usepackage{menukeys}

\renewcommand\familydefault{\sfdefault}

\lstset{basicstyle=\ttfamily,
  showstringspaces=false,
  commentstyle=\color{gray},
  keywordstyle=\color{blue},
  stringstyle=\color{teal},
  captionpos=b
}

\title{Linux w systemach wbudowanych -- Laboratorium 2}
\author{Bartłomiej Dach}

\newcommand{\centeredmenu}[1]{
	\begin{center}
		\menu{#1}
	\end{center}
}

\begin{document}

\makeatletter
\begin{flushright}
	Warszawa, \@date
\end{flushright}
\begin{center}
	\LARGE{\@title}
\end{center}
\vspace{0.25cm}
Student: \@author
\makeatother

\section{Treść zadania}

Celem drugiego ćwiczenia było przygotowanie aplikacji w języku C, obsługującą dostarczoną płytkę z przyciskami oraz diodami LED za pośrednictwem interfejsu GPIO (ang. \emph{General-Purpose Input/Output}). Aplikacja ta miała reagować na zmiany stanu przycisków za pomocą przerwań (bez czekania aktywnego).

Stworzona aplikacja to klon gry elektronicznej \emph{Simon}. W tej grze, w kolejnych rundach użytkownikowi pokazywane są coraz dłuższe sekwencje sygnałów świetlnych. Celem gracza jest bezbłędne powtórzenie danych sekwencji, wciskając odpowiednie przyciski. Dla prostoty w aplikacji ustalono limit długości tych sekwencji na 5. Podczas drugiej sesji laboratoryjnej dodane zostało ponadto utrudnienie w postaci wzrastającej szybkości pokazywania kolejnych sekwencji.

Poza utworzeniem aplikacji, w skład ćwiczenia laboratoryjnego wchodziło utworzenie pakietu Buildroota zawierającego ją, oraz przetestowanie zdalnego debugowania aplikacji za pomocą \verb+gdbserver+.

\section{Procedura odtworzenia projektu z załączonego archiwum}

W załączonym archiwum znajduje się wykonywalny skrypt powłoki o nazwie \verb+install.sh+. Po wypakowaniu archiwum i uruchomieniu skryptu z wypakowanego folderu komendą
\begin{lstlisting}[language=bash]
$ ./install.sh BR2_ROOT_PATH
\end{lstlisting}
gdzie \verb+BR2_ROOT_PATH+ oznacza katalog główny Buildroota w wersji 2016.11.2, rozpocznie się proces kopiowania odpowiednich plików oraz stosowanie łatek plików konfiguracyjnych Buildroota, a następnie kompilacji obrazu.

\section{Opis rozwiązania}

\subsection{Aplikacja \texttt{simon}}

Źródło utworzonej aplikacji znajduje się w folderze \verb+simon+ w załączonym archiwum. Jest to program w języku C, używający do obsługi pinów GPIO wbudowany w system Linux interfejs \verb+sysfs+. Użycie go sprowadza się do wykonywania operacji na plikach specjalnych udostępnionych przez sterowniki jądra.

Stworzony program jest dość nieskomplikowany w swoim działaniu. Na początku, funkcje \verb+setup_leds+ i \verb+setup_switches+ wpisują do pliku \verb+/sys/class/gpio/export+ numery pinów odpowiadające diodom LED oraz przyciskom na dostarczonej płytce, co powoduje pojawienie się katalogów o nazwie \verb+/sys/class/gpio/gpioNN+, gdzie \verb+NN+ to numer pinu.

Po wyeksportowaniu pinów są one odpowiednio konfigurowane:
\begin{itemize}
	\item W przypadku diod LED kierunek (\verb+direction+) ustalany jest na \verb+out+ -- piny te stanowią wyjście. Pozwala to na dokonywanie zapisów do pliku \verb+value+.
	\item Dla przycisków kierunek ustalany jest na \verb+in+. Dodatkowo, aby można było użyć funkcji \verb+poll+ do obsługi przerwań, do pliku \verb+edge+ wpisywana jest wartość \verb+both+.
\end{itemize}

Później losowana w funkcji \verb+generate_sequence+ jest sekwencja, która będzie pokazywana użytkownikowi. W pętli następuje naprzemienne pokazywanie początkowego podciągu wylosowanej sekwencji i wczytywanie wejścia z przycisków; długość podciągu zwiększa się o 1 w każdej iteracji pętli. W każdej iteracji interwał między kolejnymi błyśnięciami diod jest coraz mniejszy.

Najbardziej skomplikowaną częścią programu jest detekcja wciśnięć przycisków. Jest to spowodowane koniecznością wykonania \emph{debouncingu} -- wyeliminowania drgań sygnału tuż po wciśnięciu lub puszczeniu przycisku.

\begin{lstlisting}[language=C,caption=Detekcja wciśnięć pojedynczego przycisku -- funkcja \texttt{read\_button}]
#define read_from_start(device, buf, size) do {\
	if (lseek(device.fd, 0, SEEK_SET) < 0) ERR("cannot seek to start of file");\
	if (read(device.fd, buf, size) < 0) ERR("cannot read value file");\
} while (0)
// ...
int read_button(gpio_device *switches)
{
	char val;
	// dummy read and seek before polling
	for (unsigned i = 0; i < SWITCH_COUNT; ++i)
		read_from_start(switches[i], &val, sizeof(val));

	struct pollfd fds[SWITCH_COUNT];
	for (int i = 0; i < SWITCH_COUNT; ++i)
	{
		fds[i].fd = switches[i].fd;
		fds[i].events = POLLPRI | POLLERR;
	}

	if (poll(fds, SWITCH_COUNT, -1) < 0)
		ERR("poll failed");

	for (unsigned i = 0; i < SWITCH_COUNT; ++i)
		read_from_start(switches[i], &val, sizeof(val));

	switch (poll(fds, SWITCH_COUNT, 50)) {
		case -1:
			ERR("second poll failed");
		case 0:
			// no change since last poll
			break;
		default:
			// something changed, so ignore
			return -1;
	}

	int pressed = -1;
	for (unsigned i = 0; i < SWITCH_COUNT; ++i) {
		read_from_start(switches[i], &val, sizeof(val));
		if (val == '0')
			pressed = pressed < 0 ? i : -1;
	}
	return pressed;
}
\end{lstlisting}

W tym celu wykonywane są dwa wywołania funkcji \verb+poll+. Pierwsze wywołanie blokuje program do momentu wykrycia pierwszej zmiany stanu jednego z deskryptorów związanych z przyciskami. Po odblokowaniu programu, wykonywane jest drugie wywołanie -- tym razem z \emph{time-out}em ustawionym na ok. 100 ms.

\begin{itemize}
	\item Jeśli \verb+poll+ zwróci liczbę dodatnią, znaczy to, że przed upłynięciem 100 ms nastąpiła następna zmiana stanu przycisku; obydwa zdarzenia zmiany sygnału są wówczas ignorowane.
	\item Jeśli zmiana nie nastąpiła (zwrócone zostało 0), oznacza, że stan wejść ustalił się i można bezpiecznie wczytać stan wyprowadzeń.
\end{itemize}

Z przyczyn technicznych między wywołaniami funkcji \verb+poll+ następuje przesunięcie się na początek pliku funkcją \verb+lseek+ oraz wczytanie wartości, aby drugie wywołanie nie skończyło się natychmiast po pierwszym.

\subsection{Pakiet Buildroota}

Przygotowanie pakietu Buildroota przebiegało bardzo podobnie jak w przypadku skryptu Ruby w poprzednim ćwiczeniu. Dodany został katalog \verb+package/simon+ z plikami \verb+Config.in+ oraz \verb+simon.mk+, oraz wpis do pliku \verb+package/Config.in+ wskazujący na nową paczkę. Plik \verb+.mk+ został poszerzony o komendę, która odpowiada za zbudowanie pliku wykonywalnego ze źródeł:
\begin{lstlisting}[language=bash,caption=Zawartość pliku \texttt{simon.mk}]
################################################################################
#
# simon
#
################################################################################

SIMON_VERSION = 1.4
SIMON_SITE = $(TOPDIR)/../simon
SIMON_SITE_METHOD = local

define SIMON_BUILD_CMDS
   $(MAKE) $(TARGET_CONFIGURE_OPTS) CFLAGS+="-std=gnu99" simon -C $(@D)
endef
define SIMON_INSTALL_TARGET_CMDS 
   $(INSTALL) -D -m 0755 $(@D)/simon $(TARGET_DIR)/usr/bin 
endef
SIMON_LICENSE = MIT

$(eval $(generic-package))
\end{lstlisting}
W celu umożliwienia korzystania z flagi \verb+-std=gnu99+, po ustawieniu opcji poprzez zmienną \\ 
\verb+TARGET_CONFIGURE_OPTS+ zostaje ona dołączona do zmiennej środowiskowej \verb+CFLAGS+.

\subsection{Przygotowanie obrazu}

Po przygotowaniu pakietu, skompilowany został obraz systemu, z następującymi standardowymi opcjami:

\begin{itemize}
	\item port \verb+getty+ został ustawiony na \verb+ttyAMA0+,
	\item główna lokalizacja pobierania została ustawiona na serwer lustrzany w sieci lokalnej o adresie \verb+http://192.168.137.24/dl+,
	\item wyłączone zostało wsparcie dla akceleracji operacji zmiennoprzecinkowych,
	\item użyty został zewnętrzny \emph{toolchain} Sourcery CodeBench ARM 2014.05,
	\item obraz systemu został skompilowany z systemem plików \verb+initramfs+ oraz skompresowany za pomocą gzip.
\end{itemize}
Ponadto do systemu została dodana aplikacja \verb+simon+, poprzez wybranie opcji
\centeredmenu{Target packages > Games > simon}
Aby umożliwić wykonanie zdalnego debugowania poprzez \verb+gdb+, wybrane zostały również opcje
\centeredmenu{Target packages > Debugging, profiling and benchmark > gdb}
Po wybraniu tej opcji, automatycznie zaznaczana jest opcja \verb+gdbserver+.

Aby komputer-host mógł połączyć się ze zdalnym serwerem, należy włączyć budowanie kompatybilnej z nim wersji \verb+gdb+ opcją
\centeredmenu{Toolchain > Build cross gdb for the host}
Wreszcie, zaznaczona została opcja
\centeredmenu{Build options > build packages with debugging symbols}
Po jej zaznaczeniu, Buildroot w trakcie kompilacji skompiluje zaznaczone pakiety dwukrotnie -- raz na system docelowy, bez symboli, oraz drugi, z opcją \verb+-g+, zawierający symbole. Pliki wykonywalne powstałe z drugiej kompilacji można wtedy podać jako argument do zdalnego wywołania \verb+gdb+.

\subsection{Zdalne debugowanie za pomocą \texttt{gdbserver}}

Po wykonaniu obrazu, skopiowaniu go na kartę SD i zrestartowaniu płytki przystąpiono do włączenia zdalnego debugowania. Najpierw, na płytce została uruchomiona instancja serwera komendą
\begin{lstlisting}
# gdbserver host:7777 simon
\end{lstlisting}
zaś na stacji roboczej połączono się z serwerem komendami
\begin{lstlisting}
$ /malina/dachb/buildroot-2016.11.2/output/host/opt/ext-toolchain/bin/\
  arm-none-linux-gnueabi-gdb /malina/dachb/buildroot-2016.11.2/output/build/\
  simon-1.4/simon
(gdb) target remote 192.168.143.149:7777
\end{lstlisting}
W wyniku powyższych czynności można debugować kod wykonywany na płytce z poziomu komputera laboratoryjnego, wykonując takie operacje wspierane przez \verb+gdb+, jak:
\begin{itemize}
	\item ustawianie breakpointów (\emph{break}),
	\item przejście do następnej instrukcji (\emph{next}, \emph{step}),
	\item kontynuacja wykonania programu do następnego breakpointa (\emph{continue}),
	\item wypisanie wartości jednej z zadeklarowanych zmiennych (\emph{print}).
\end{itemize}

\end{document}
