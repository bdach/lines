\documentclass[10pt,a4paper]{article}
\usepackage[T1]{fontenc}
\usepackage[utf8]{inputenc}
\usepackage{graphicx}
\usepackage{tabularx}
\usepackage{helvet}
\usepackage{bera}
\usepackage{hyperref}
\usepackage[a4paper,margin=1in]{geometry}
\usepackage[polish]{babel}
\usepackage{float}
\usepackage{listings}
\usepackage{menukeys}

\renewcommand\familydefault{\sfdefault}

\lstset{basicstyle=\ttfamily,
  showstringspaces=false,
  commentstyle=\color{gray},
  keywordstyle=\color{blue},
  stringstyle=\color{teal},
  captionpos=b
}

\title{Linux w systemach wbudowanych -- Laboratorium 5}
\author{Bartłomiej Dach}

\newcommand{\centeredmenu}[1]{
	\begin{center}
		\menu{#1}
	\end{center}
}

\lstdefinelanguage{javascript}{
	keywords={typeof, new, true, false, catch, function, return, null, catch, switch, var, if, in, while, do, else, case, break},
	ndkeywords={class, export, boolean, throw, implements, import, this},
	sensitive=false,
	comment=[l]{//},
	morecomment=[s]{/*}{*/},
	morestring=[b]',
	morestring=[b]"
}

\lstdefinelanguage{diff}{
	basicstyle=\ttfamily\small,
	morecomment=[f][\color{blue}]{@@},
	morecomment=[f][\color{teal}]{+},
	morecomment=[f][\color{red}]{-},
}

\begin{document}

\makeatletter
\begin{flushright}
	Warszawa, \@date
\end{flushright}
\begin{center}
	\LARGE{\@title}
\end{center}
\vspace{0.25cm}
Student: \@author
\makeatother

\section{Treść zadania}

Celem piątego zadania laboratoryjnego było przeniesienie rozwiązania zadania 4. zrealizowanego
za pomocą platformy Buildroot na jedną z konkurencyjnych platform, takich, jak OpenWrt
\cite{openwrt}, LEDE Project \cite{lede} czy Yocto Project \cite{yocto}.

W ramach zadania 4. zrealizowałem szafę grającą odtwarzającą pliki MP3, sterowaną za pomocą
interfejsów: sieciowego oraz przycisków na dołączonej płytce. Rozwiązanie to zostanie przeniesione
w zadaniu 5. na system dostarczony przez LEDE Project.

\section{Procedura odtworzenia projektu z załączonego archiwum}

% też jeszcze nie wiem

\section{Opis rozwiązania}

\subsection{Instalacja prekompilowanego obrazu LEDE Project}

% nie wiem

\subsection{Zmiany w aplikacji}

Pierwotna aplikacja do odtwarzania muzyki używała programu VLC media player oraz dostarczonych
przez jego twórców bindingów do języka Python. Z racji braku VLC w liście paczek LEDE Project
zastosowany został Music Player Daemon \cite{mpd} łącznie z bindingami do Pythona w postaci
modułu python-mpd2 \cite{python-mpd2}.

Obsługa przycisków za pomocą biblioteki RPi.GPIO okazała się niemożliwa do wykonania
na prekompilowanym obrazie z racji braku m.in. aplikacji \texttt{ccache} lub jej alternatywy
w postaci kompilatora \texttt{gcc}, nagłówków deweloperskich Pythona 3 oraz braku plików
specjalnych \texttt{/dev/mem}. W związku z tymi problemami obsługa przycisków została zrealizowana
za pośrednictwem interfejsu \texttt{sysfs} przy pomocy biblioteki \texttt{gpio} \cite{gpio}.

\subsection{Konfiguracja systemu}

Po zainstalowaniu gotowego obrazu na karcie SD i restarcie systemu prekompilowany system Linux
jest gotowy do użycia. Na początku, w celu umożliwienia komunikacji za pośrednictwem wbudowanego
serwera SSH oraz korzystania z menedżera pakietów \texttt{opkg} zmodyfikowane zostały ustawienia
sieci tak, aby włączyć protokół DHCP w celu uzyskania adresu IP:

\begin{lstlisting}[caption=Włączanie klienta DHCP na interfejsie \texttt{lan}]
# uci set network.lan.proto=dhcp
# uci commit
# /etc/init.d/network restart
\end{lstlisting}

Po ustanowieniu połączenia przystąpiono do instalacji zależności przenoszonej aplikacji po
uprzednim zaktualizowaniu listy zależności poleceniem \texttt{opkg update}:

\begin{enumerate}
	\item Python 3: \texttt{opkg install python3},
	\item Menedżer modułów Pythona -- \texttt{pip}: \texttt{opkg install python3-pip},
	\item Z bliżej nieokreślonych powodów konieczna po instalacji programu \texttt{pip3}
		była również aktualizacja zestawu narzędzi \texttt{setuptools} poleceniem
\begin{lstlisting}
# pip3 install --upgrade setuptools
\end{lstlisting}
	\item Instalacja modułów Pythona używanych w aplikacji:
\begin{lstlisting}
# pip3 install flask
# pip3 install gpio
# pip3 install python-mpd2
\end{lstlisting}
	\item Instalacja serwera oraz klienta \texttt{mpd}:
\begin{lstlisting}
# opkg install mpc
# opkg install mpd-mini
\end{lstlisting}
		W przypadku korzystania z wersji migawkowej (ang. \texttt{snapshot}) systemu, aby
		dodać ten pakiet, konieczne było dodanie kanału (ang. \texttt{feed})
\begin{lstlisting}
src/gz release-packages 
	http://downloads.lede-project.org/releases/
	17.01.1/packages/arm_arm1176jzf-s_vfp/packages
\end{lstlisting}
		do listy kanałów, z których pobierane są informacje o paczkach.
\end{enumerate}

% overlay???

\begin{thebibliography}{9}

	\bibitem{flask}
		Flask: A Python Microframework,
		\url{http://flask.pocoo.org/},
		BSD

	\bibitem{gpio}
		gpio,
		GitHub,
		\url{https://github.com/vitiral/gpio}

	\bibitem{lede}
		LEDE project,
		\url{https://lede-project.org/}

	\bibitem{materialize}
		Materialize CSS,
		\url{http://materializecss.com/},
		MIT

	\bibitem{mpd}
		Music Player Daemon,
		\url{https://www.musicpd.org/},
		GPL

	\bibitem{openwrt}
		OpenWrt,
		\url{https://openwrt.org/}
	
	\bibitem{python-mpd2}
		python-mpd2
		GitHub,
		\url{https://github.com/Mic92/python-mpd2},
		LGPL-3.0

	\bibitem{yocto}
		Yocto Project,
		\url{https://www.yoctoproject.org/}

\end{thebibliography}

\end{document}
