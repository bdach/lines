\documentclass[10pt,a4paper]{article}
\usepackage[T1]{fontenc}
\usepackage[utf8]{inputenc}
\usepackage{graphicx}
\usepackage{tabularx}
\usepackage{helvet}
\usepackage{bera}
\usepackage{hyperref}
\usepackage[a4paper,margin=1in]{geometry}
\usepackage[polish]{babel}
\usepackage{float}
\usepackage{listings}
\usepackage{menukeys}

\renewcommand\familydefault{\sfdefault}

\lstset{basicstyle=\ttfamily,
  showstringspaces=false,
  commentstyle=\color{gray},
  keywordstyle=\color{blue},
  stringstyle=\color{teal},
  captionpos=b
}

\title{Linux w systemach wbudowanych -- Laboratorium 5}
\author{Bartłomiej Dach}

\newcommand{\centeredmenu}[1]{
	\begin{center}
		\menu{#1}
	\end{center}
}

\lstdefinelanguage{javascript}{
	keywords={typeof, new, true, false, catch, function, return, null, catch, switch, var, if, in, while, do, else, case, break},
	ndkeywords={class, export, boolean, throw, implements, import, this},
	sensitive=false,
	comment=[l]{//},
	morecomment=[s]{/*}{*/},
	morestring=[b]',
	morestring=[b]"
}

\lstdefinelanguage{diff}{
	basicstyle=\ttfamily\small,
	morecomment=[f][\color{blue}]{@@},
	morecomment=[f][\color{teal}]{+},
	morecomment=[f][\color{red}]{-},
}

\begin{document}

\makeatletter
\begin{flushright}
	Warszawa, \@date
\end{flushright}
\begin{center}
	\LARGE{\@title}
\end{center}
\vspace{0.25cm}
Student: \@author
\makeatother

\section{Treść zadania}

\section{Procedura odtworzenia projektu z załączonego archiwum}

\section{Opis rozwiązania}

\begin{thebibliography}{9}

\bibitem{flask}
	Flask: A Python Microframework,
		\url{http://flask.pocoo.org/},
		BSD

\bibitem{materialize}
	Materialize CSS,
		\url{http://materializecss.com/},
		MIT

\bibitem{rpi-gpio}
	raspberry-gpio-python,
		SourceForge,
		\url{https://sourceforge.net/projects/raspberry-gpio-python/},
		MIT

\end{thebibliography}

\end{document}
