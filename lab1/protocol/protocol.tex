\documentclass[10pt,a4paper]{article}
\usepackage[T1]{fontenc}
\usepackage[utf8]{inputenc}
\usepackage{graphicx}
\usepackage{tabularx}
\usepackage{helvet}
\usepackage{inconsolata}
\usepackage{hyperref}
\usepackage[a4paper,margin=1in]{geometry}
\usepackage[polish]{babel}
\usepackage{float}
\usepackage{listings}
\usepackage{menukeys}

\renewcommand\familydefault{\sfdefault}

\lstset{basicstyle=\ttfamily,
  showstringspaces=false,
  commentstyle=\color{grey},
  keywordstyle=\color{blue}
}

\title{Linux w systemach wbudowanych -- Laboratorium 1}
\author{Bartłomiej Dach}

\newcommand{\centeredmenu}[1]{
	\begin{center}
		\menu{#1}
	\end{center}
}

\begin{document}

\makeatletter
\begin{flushright}
	Warszawa, \@date
\end{flushright}
\begin{center}
	\LARGE{\@title}
\end{center}
\vspace{0.25cm}
Student: \@author
\makeatother

\section{Treść zadania}

W ciągu pierwszej sesji laboratoryjnej do wykonania zostały wyznaczone następujące czynności:

\begin{enumerate}
	\item Uruchomienie płytki Raspberry Pi w trybie ratunkowym oraz zapoznanie się z możliwościami tego trybu.
	\item Przygotowanie obrazu systemu Linux z systemem plików \verb+initramfs+, spełniającym następujące wymagania:
	\begin{enumerate}
		\item Raspberry Pi automatycznie łączy się z siecią, pobierając konfigurację za pośrednictwem DHCP.
			Połączenie jest zestawiane automatycznie w momencie podłączenia kabla sieciowego, oraz automatycznie
			zrywane w momencie jego odłączenia.
		\item Nazwa hosta systemu powinna mieć postać \verb+imie_nazwisko+ autora.
		\item Przy starcie systemu, zegar systemowy powinien być synchronizowany z wybranym serwerem NTP.
		\item Obraz systemu zawiera interpreter języka Ruby oraz przykładowy pakiet Buildroota, zawierający przykładowy
			skrypt w Ruby. Skrypt ten powinien być uruchamiany automatycznie przy starcie systemu.
		\item W systemie powinien obok użytkownika \verb+root+ istnieć jeszcze jeden użytkownik o dowolnej nazwie.
			Obaj użytkownicy powinni mieć ustalone hasła.
	\end{enumerate}
\end{enumerate}

\section{Procedura odtworzenia projektu z załączonego archiwum}

\section{Opis rozwiązania}

\subsection{Czynności wstępne}

Na początku ćwiczenia płytka uruchomiona została w trybie ratunkowym poprzez wciśnięcie dowolnego klawisza w czasie oczekiwania bootloadera \verb+uboot+ oraz wpisanie komendy \verb+run rescue+. Następnie sprawdzono możliwości systemu ratunkowego, w szczególności możliwość skopiowania bądź usunięcia własnego obrazu systemu, oraz możliwość połączenia się przez SSH z komputerami laboratoryjnymi.

Następnie na komputery laboratoryjne pobrany został Buildroot w wersji 2016.11.2. Konfigurację jądra rozpoczęto od załadowania domyślnej konfiguracji dla płytki Raspberry Pi poleceniem
\begin{lstlisting}[language=bash]
$ make raspberrypi_defconfig
\end{lstlisting}
Następnie w celu umożliwienia używania prekompilowanych narzędzi (ang. \emph{toolchain}), wyłączono obsługę sprzętową operacji zmiennoprzecinkowych, zmieniając opcję
\centeredmenu{Target options > Target ABI}
z \verb+EABIhf+ na \verb+EABI+. Następnie włączono generowanie systemu plików \verb+initramfs+ poprzez zaznaczenie opcji
\centeredmenu{Filesystem images > initial RAM filesystem linked into linux kernel}
Dodatkowo, w celu wykonywania modyfikacji plików konfiguracyjnych i skryptów uruchomieniowych na systemie docelowym, skonfigurowana została nakładka (ang. \emph{overlay}) na system plików, za pomocą opcji
\centeredmenu{System configuration > Root filesystem overlay directories}
Podana została ścieżka wskazująca na folder \verb+overlay/+ umieszczony równolegle z katalogiem domowym Buildroota.

\subsection{Automatyczna konfiguracja sieci}

W celu automatycznej konfiguracji interfejsów sieciowych za pośrednictwem DHCP został użyty daemon \verb+ifplugd+. W celu jego instalacji najpierw została dodana jego zależność, oznaczona symbolem \verb+BR2_PACKAGE_BUSYBOX_SHOW_OTHERS+, pozwalająca na wyświetlenie dodatkowych paczek dostarczanych przez \verb+busybox+:
\centeredmenu{Target packages > BusyBox > Show packages that are also provided by busybox}
a następnie sam pakiet zawierający \verb+ifplugd+:
\centeredmenu{Target packages > Networking applications > ifplugd}
Po dodaniu pakietu, automatycznie dodawany jest również \emph{init script} uruchamiający daemon przy starcie systemu. \verb+ifplugd+ nasłuchuje zmiany w stanie połączenia sieciowego i automatycznie włącza/wyłącza interfejs, używając DHCP do automatycznej konfiguracji adresów sieciowych.

\subsection{Ustawienie nazwy hosta}

Zmiana nazwy hosta skompilowanego systemu została wykonana poprzez modyfikację opcji
\centeredmenu{System configuration > System hostname}
i ustawienie jej wartości na ciąg znaków \verb+bartlomiej_dach+.

\subsection{Synchronizacja zegara systemowego}

W celu możliwości synchronizacji zegara systemowego, najpierw wybrane została paczka Buildroota umożliwiające korzystanie z protokołu NTP (ang. \emph{Network Time Protocol}) -- \verb+ntpd+, będący daemonem-klientem NTP:
\centeredmenu{Target packages > Networking applications > ntp > ntpd}
Razem z daemonem \verb+ntpd+ dodawany jest \emph{init script} uruchamiający go podczas startu systemu. Aby daemon ten korzystał z polskiej puli serwerów, w nakładce dodany został plik \verb+/etc/ntp.conf+ następującej treści:
\begin{lstlisting}[caption=Zawartość pliku \texttt{ntp.conf}]
server 0.pl.pool.ntp.org
server 1.pl.pool.ntp.org
server 2.pl.pool.ntp.org
server 3.pl.pool.ntp.org
\end{lstlisting}

\subsection{Interpreter języka Ruby oraz własny skrypt jako paczka}

W celu dodania do obrazu interpretera języka skryptowego Ruby dodana została paczka
\centeredmenu{Target packages > Interpreter languages and scripting > ruby}
po uprzednim zaznaczeniu wymaganej zależności obsługi ,,szerokich znaków'' (ang. \emph{wchar}, \emph{wide character})
\centeredmenu{Toolchain > Enable WCHAR support}
% TODO: Making a package and adding an init script

\subsection{Dodanie użytkownika oraz haseł}

Hasło użytkownika \verb+root+ zostało ustawione za pośrednictwem opcji
\centeredmenu{System configuration > Root password}
Dodawanie innych użytkowników w środowisku Buildroot odbywa się za pomocą tzw. tablic użytkowniów (ang. \emph{users tables}). Ścieżkę do pliku tekstowego zawierającego taką tablicę można ustawić w opcji
\centeredmenu{System configuration > Path to the users tables}
% TODO: contents and syntax of the users tables

\end{document}
