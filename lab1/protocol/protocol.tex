\documentclass[10pt,a4paper]{article}
\usepackage[T1]{fontenc}
\usepackage[utf8]{inputenc}
\usepackage{graphicx}
\usepackage{tabularx}
\usepackage{helvet}
\usepackage{inconsolata}
\usepackage{hyperref}
\usepackage[a4paper,margin=1in]{geometry}
\usepackage[polish]{babel}
\usepackage{float}
\usepackage{listings}
\usepackage{menukeys}

\renewcommand\familydefault{\sfdefault}

\lstset{basicstyle=\ttfamily,
  showstringspaces=false,
  commentstyle=\color{gray},
  keywordstyle=\color{blue},
  stringstyle=\color{teal},
  captionpos=b
}

\title{Linux w systemach wbudowanych -- Laboratorium 1}
\author{Bartłomiej Dach}

\newcommand{\centeredmenu}[1]{
	\begin{center}
		\menu{#1}
	\end{center}
}

\begin{document}

\makeatletter
\begin{flushright}
	Warszawa, \@date
\end{flushright}
\begin{center}
	\LARGE{\@title}
\end{center}
\vspace{0.25cm}
Student: \@author
\makeatother

\section{Treść zadania}

W ciągu pierwszej sesji laboratoryjnej do wykonania zostały wyznaczone następujące czynności:

\begin{enumerate}
	\item Uruchomienie płytki Raspberry Pi w trybie ratunkowym oraz zapoznanie się z możliwościami tego trybu.
	\item Przygotowanie obrazu systemu Linux z systemem plików \verb+initramfs+, spełniającym następujące wymagania:
	\begin{enumerate}
		\item Raspberry Pi automatycznie łączy się z siecią, pobierając konfigurację za pośrednictwem DHCP.
			Połączenie jest zestawiane automatycznie w momencie podłączenia kabla sieciowego, oraz automatycznie
			zrywane w momencie jego odłączenia.
		\item Nazwa hosta systemu powinna mieć postać \verb+imie_nazwisko+ autora.
		\item Przy starcie systemu, zegar systemowy powinien być synchronizowany z wybranym serwerem NTP.
		\item Obraz systemu zawiera interpreter języka Ruby oraz przykładowy pakiet Buildroota, zawierający przykładowy
			skrypt w Ruby. Skrypt ten powinien być uruchamiany automatycznie przy starcie systemu.
		\item W systemie powinien obok użytkownika \verb+root+ istnieć jeszcze jeden użytkownik o dowolnej nazwie.
			Obaj użytkownicy powinni mieć ustalone hasła.
	\end{enumerate}
\end{enumerate}

\section{Procedura odtworzenia projektu z załączonego archiwum}

W załączonym archiwum znajduje się wykonywalny skrypt powłoki o nazwie \verb+install.sh+. Po uruchomieniu go komendą
\begin{lstlisting}[language=bash]
$ ./install.sh BR2_ROOT_PATH
\end{lstlisting}
gdzie \verb+BR2_ROOT_PATH+ oznacza katalog nadrzędny Buildroota w wersji 2016.11.2, rozpocznie się proces kopiowania odpowiednich plików oraz aplikowania łatek na pliki konfiguracyjne Buildroota, zakończony kompilacją gotowego obrazu, który wystarczy umieścić na płytce.

\section{Opis rozwiązania}

\subsection{Czynności wstępne}

Na początku ćwiczenia płytka uruchomiona została w trybie ratunkowym poprzez wciśnięcie dowolnego klawisza w czasie oczekiwania bootloadera \verb+uboot+ oraz wpisanie komendy \verb+run rescue+. Następnie sprawdzono możliwości systemu ratunkowego, w szczególności możliwość skopiowania bądź usunięcia własnego obrazu systemu, oraz możliwość połączenia się przez SSH z komputerami laboratoryjnymi.

Następnie na komputery laboratoryjne pobrany został Buildroot w wersji 2016.11.2. Konfigurację jądra rozpoczęto od załadowania domyślnej konfiguracji dla płytki Raspberry Pi poleceniem
\begin{lstlisting}[language=bash]
$ make raspberrypi_defconfig
\end{lstlisting}
Następnie w celu umożliwienia używania prekompilowanych narzędzi (ang. \emph{toolchain}), wyłączono obsługę sprzętową operacji zmiennoprzecinkowych, zmieniając opcję
\centeredmenu{Target options > Target ABI}
z \verb+EABIhf+ na \verb+EABI+. Następnie włączono generowanie systemu plików \verb+initramfs+ poprzez zaznaczenie opcji
\centeredmenu{Filesystem images > initial RAM filesystem linked into linux kernel}
Dodatkowo, w celu wykonywania modyfikacji plików konfiguracyjnych i skryptów uruchomieniowych na systemie docelowym, skonfigurowana została nakładka (ang. \emph{overlay}) na system plików, za pomocą opcji
\centeredmenu{System configuration > Root filesystem overlay directories}
Podana została ścieżka wskazująca na folder \verb+overlay/+ umieszczony równolegle z katalogiem domowym Buildroota.

\subsection{Automatyczna konfiguracja sieci}

W celu automatycznej konfiguracji interfejsów sieciowych za pośrednictwem DHCP został użyty daemon \verb+ifplugd+. W celu jego instalacji najpierw została dodana jego zależność, oznaczona symbolem \verb+BR2_PACKAGE_BUSYBOX_SHOW_OTHERS+, pozwalająca na wyświetlenie dodatkowych paczek dostarczanych przez \verb+busybox+:
\centeredmenu{Target packages > BusyBox > Show packages that are also provided by busybox}
a następnie sam pakiet zawierający \verb+ifplugd+:
\centeredmenu{Target packages > Networking applications > ifplugd}
Po dodaniu pakietu, automatycznie dodawany jest również \emph{init script} uruchamiający daemon przy starcie systemu. \verb+ifplugd+ nasłuchuje zmiany w stanie połączenia sieciowego i automatycznie włącza/wyłącza interfejs, używając DHCP do automatycznej konfiguracji adresów sieciowych.

\subsection{Ustawienie nazwy hosta}

Zmiana nazwy hosta skompilowanego systemu została wykonana poprzez modyfikację opcji
\centeredmenu{System configuration > System hostname}
i ustawienie jej wartości na ciąg znaków \verb+bartlomiej_dach+.

\subsection{Synchronizacja zegara systemowego}

W celu możliwości synchronizacji zegara systemowego, najpierw wybrane została paczka Buildroota umożliwiające korzystanie z protokołu NTP (ang. \emph{Network Time Protocol}) -- \verb+ntpd+, będący daemonem-klientem NTP:
\centeredmenu{Target packages > Networking applications > ntp > ntpd}
Razem z daemonem \verb+ntpd+ dodawany jest \emph{init script} uruchamiający go podczas startu systemu. Aby daemon ten korzystał z polskiej puli serwerów, w nakładce dodany został plik \verb+/etc/ntp.conf+ następującej treści:
\begin{lstlisting}[caption=Zawartość pliku \texttt{ntp.conf}]
server 0.pl.pool.ntp.org
server 1.pl.pool.ntp.org
server 2.pl.pool.ntp.org
server 3.pl.pool.ntp.org
\end{lstlisting}

\subsection{Interpreter języka Ruby oraz własny skrypt jako paczka}

W celu dodania do obrazu interpretera języka skryptowego Ruby dodana została paczka
\centeredmenu{Target packages > Interpreter languages and scripting > ruby}
po uprzednim zaznaczeniu wymaganej zależności obsługi ,,szerokich znaków'' (ang. \emph{wchar}, \emph{wide character})
\centeredmenu{Toolchain > Enable WCHAR support}
Następnie przystąpiono do stworzenia własnej paczki ze skryptem. Plik ze skryptem znajduje się w katalogu \verb+myrubyscript+ i jest wariantem \emph{Hello world} w języku Ruby:
\begin{lstlisting}[language=ruby, caption=Zawartość skryptu \texttt{hello.rb}]
#!/usr/bin/ruby
puts 'Hello world!'
\end{lstlisting}
W katalogu \verb+package+ Buildroota utworzony został nowy folder dla tworzonej paczki o nazwie \verb+myrubyscript+. W tym folderze dodane zostały dwa pliki konfiguracyjne: \verb+Config.in+ określający nazwę dodawanej opcji konfiguracyjnej kompilowanego jądra i szczegóły paczki, oraz \verb+myrubyscript.mk+ określający sposób instalacji paczki na kompilowanym jądrze.
\begin{lstlisting}[caption=Zawartość pliku \texttt{Config.in}]
config BR2_PACKAGE_MYRUBYSCRIPT
	bool "myrubyscript"
	depends on BR2_PACKAGE_RUBY
	help
	  Example Ruby script added to Buildroot
	  for exercise purposes.

	  https://www.ruby-lang.org/

comment "myrubyscript needs ruby"
	depends on !BR2_PACKAGE_RUBY
\end{lstlisting}
W pliku \verb+Config.in+ określono zależność tworzonej paczki od interpretera Ruby oraz dodano komentarz podpowiadający, jak wybrać ją, jeśli interpreter nie został wybrany przez użytkownika konfigurującego jądro.
\begin{lstlisting}[language=bash,caption=Zawartość pliku \texttt{myrubyscript.mk}]
################################################################################
#
# myrubyscript
#
################################################################################

MYRUBYSCRIPT_VERSION = 1.0
MYRUBYSCRIPT_SITE = $(TOPDIR)/../myrubyscript
MYRUBYSCRIPT_SITE_METHOD = local
MYRUBYSCRIPT_DEPENDENCIES = ruby

define MYRUBYSCRIPT_INSTALL_TARGET_CMDS
	$(INSTALL) -D -m 0755 $(@D)/hello.rb $(TARGET_DIR)/usr/bin
endef
MYRUBYSCRIPT_LICENSE = MIT

$(eval $(generic-package))
\end{lstlisting}
Zgodnie z konwencją Buildroota, plik rozpczyna się od nagłówka złożonego z 80 znaków komentarza, zawierającego nazwę paczki. Określony został katalog ze źródłami oraz metoda lokalizacji źródeł, zależności paczki oraz komendy, jakie mają być wykonane, aby zainstalować pakiet (w tym przypadku, skopiowanie skryptu \verb+hello.rb+ do katalogu \verb+/usr/bin+ w obrazie docelowym, z uprawnieniami \verb+rwxr-xr-x+).

Aby dodana paczka pojawiła się w menu konfiguracyjnym jądra, edytowany zostal plik \\
\verb+$BR2_PATH/packages/Config.in+. Poniżej linii definiującej interpreter Ruby dodana została linia importująca zawartość stworzonego pliku konfiguracyjnego:
\begin{lstlisting}[caption=Modyfikacja głównego pliku \texttt{Config.in}]
	source "package/ruby/Config.in"
	source "package/myrubyscript/Config.in"
\end{lstlisting}
Umiejscowienie paczki w menu nie jest zgodne z przyjętymi standardami Buildroota, lecz dla skrócenia czasu konfiguracji i ułatwienia znalezienia dodanej paczki zastosowane zostało powyższe przejściowe rozwiązanie.

Po wykonaniu powyższych czynności pozwala na zainstalowanie paczki w katalogu \verb+/usr/bin+ systemu wynikowego. Jedyne, co pozostało zrobić, to dodanie \emph{init scriptu} uruchamiającego \verb+hello.rb+ do katalogu \verb+/etc/init.d/+ nakładki:
\begin{lstlisting}[language=bash,caption=Zawartość pliku \texttt{overlay/etc/init.d/S50myrubyscript}]
#!/bin/sh
case $1 in
	"start") hello.rb ;;
	*)
esac
exit 0
\end{lstlisting}
Po dodaniu tego pliku i uruchomieniu skompilowanego obrazu systemu, przed logowaniem pojawia się komunikat \verb+Hello world!+.

\subsection{Dodanie użytkownika oraz haseł}

Hasło użytkownika \verb+root+ zostało ustawione za pośrednictwem opcji
\centeredmenu{System configuration > Root password}
Dodawanie innych użytkowników w środowisku Buildroot odbywa się za pomocą tzw. tablic użytkowniów (ang. \emph{users tables}). Ścieżkę do pliku tekstowego zawierającego taką tablicę można ustawić w opcji
\centeredmenu{System configuration > Path to the users tables}
Aby utworzyć nowego użytkownika o nazwie \verb+dachb+ i haśle \verb+tajne+, dodano plik z tablicami o nazwie \verb+users+ o następującej treści:
\begin{lstlisting}[language=bash,caption=Zawartość pliku \texttt{users}]
# username uid groupname gid password homedir shell othergroups comment
dachb -1 users -1 =tajne - /bin/sh - Bartlomiej Dach
\end{lstlisting}
Ustawienie identyfikatorów użytkownika oraz grupy na -1 oznacza, że zostaną one wyznaczone automatycznie w procesie generowania obrazu przez Buildroot (jeśli grupa o podanej nazwie nie istnieje, zostanie automatycznie stworzona). Znak równości przed hasłem oznacza zaś, że zostało ono podane w postaci jawnej (ang. \emph{plaintext}).

\end{document}
