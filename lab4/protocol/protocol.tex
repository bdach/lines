\documentclass[10pt,a4paper]{article}
\usepackage[T1]{fontenc}
\usepackage[utf8]{inputenc}
\usepackage{graphicx}
\usepackage{tabularx}
\usepackage{helvet}
\usepackage{bera}
\usepackage{hyperref}
\usepackage[a4paper,margin=1in]{geometry}
\usepackage[polish]{babel}
\usepackage{float}
\usepackage{listings}
\usepackage{menukeys}

\renewcommand\familydefault{\sfdefault}

\lstset{basicstyle=\ttfamily,
  showstringspaces=false,
  commentstyle=\color{gray},
  keywordstyle=\color{blue},
  stringstyle=\color{teal},
  captionpos=b
}

\title{Linux w systemach wbudowanych -- Laboratorium 4}
\author{Bartłomiej Dach}

\newcommand{\centeredmenu}[1]{
	\begin{center}
		\menu{#1}
	\end{center}
}

\lstdefinelanguage{javascript}{
	keywords={typeof, new, true, false, catch, function, return, null, catch, switch, var, if, in, while, do, else, case, break},
	ndkeywords={class, export, boolean, throw, implements, import, this},
	sensitive=false,
	comment=[l]{//},
	morecomment=[s]{/*}{*/},
	morestring=[b]',
	morestring=[b]"
}

\lstdefinelanguage{diff}{
	basicstyle=\ttfamily\small,
	morecomment=[f][\color{blue}]{@@},
	morecomment=[f][\color{teal}]{+},
	morecomment=[f][\color{red}]{-},
}

\begin{document}

\makeatletter
\begin{flushright}
	Warszawa, \@date
\end{flushright}
\begin{center}
	\LARGE{\@title}
\end{center}
\vspace{0.25cm}
Student: \@author
\makeatother

\section{Treść zadania}

Celem zadania 4. było zrealizowanie przy pomocy płytki Raspberry Pi urządzenia, które jest
wyposażone w interfejs użytkownika złożony z:

\begin{itemize}
	\item przycisków oraz diod LED, pozwalających na podstawową obsługę urządzenia,
	\item interfejsu sieciowego lub WWW, udostępniających bardziej zaawansowane funkcje.
\end{itemize}

Wybranym przeze mnie tematem ćwiczenia było stworzenie szafy grającej odtwarzającej pliki
w formacie MP3. Za pomocą interfejsu WWW użytkownicy wybierają znajdujące się na karcie SD
utwory, które trafiają do kolejki odtwarzania. Mają oni również możliwość dodawania
własnych folderów oraz przesyłania własnych plików muzycznych.

Zarówno za pomocą interfejsu sieciowego, jak i przycisków, możliwa jest kontrola odtwarzania
w postaci: pauzowania/wznawiania obecnego utworu, pomijania obecnie odtwarzanego utworu
(po nim grany jest następny utwór w kolejce), oraz regulacji głośności odtwarzania.

\section{Procedura odtworzenia projektu z załączonego archiwum}

W folderze \directory{files} znajduje się skrypt instalacyjny o nazwie \directory{install.sh}.
Po uruchomieniu go poleceniem
\begin{lstlisting}[language=bash]
$ ./install.sh ${BR2_PATH}
\end{lstlisting}
gdzie \texttt{BR2\_PATH} oznacza katalog główny Buildroota, skrypt ustawi konfigurację pakietów
Buildroota, skopiuje pliki nakładki oraz rozpocznie proces kompilacji obrazu. Po zakończeniu
kompilacji stosowana jest dodatkowo łatka do wygenerowanego przez Buildroot drzewa urządzeń.

Na system docelowy należy skopiować drzewo urządzeń (\texttt{bcm2708-rpi-b.dtb}) oraz obraz
systemu (\texttt{zImage}) z katalogu \texttt{output/images}.

\section{Opis rozwiązania}

\subsection{Konfiguracja obrazu}

Na początku konfiguracji ustawiono standardowe opcje, zawarte w poradniku laboratoryjnym
(ustawienia toolchaina, serwer lustrzany, TTY, opcje obrazu systemu).

Na system plików dodana została nakładka, w celu dołączenia niezbędnych plików. W nakładce
znajdują się: pliki serwera, skrypt uruchomieniowy szafy grającej oraz plik
\texttt{/etc/inittab} zmodyfikowany tak, aby trzecia partycja karty SD była montowana pod
katalog \texttt{/music} w systemie docelowym.

Do systemu dodane zostały paczki:

\begin{itemize}
	\item ALSA -- do kontroli oraz obsługi urządzeń audio:
		\centeredmenu{Target packages > Libraries > Audio/sound > \texttt{alsa-lib}}
		\centeredmenu{Target packages > Audio and video applications > \texttt{alsa-utils}}
	\item ffmpeg -- do dekodowania formatu MP3,
		\centeredmenu{Target packages > Audio and video applications > \texttt{ffmpeg}}
	\item vlc -- do właściwego odtwarzania plików,
		\centeredmenu{Target packages > Audio and video applications > \texttt{vlc}}
	\item interpreter języka Python,
		\centeredmenu{Target packages > Interpreter languages and scripting > \texttt{python3}}
	\item mikroframework Flask razem z automatycznie zaznaczonymi zależnościami,
		\centeredmenu{Target packages > Interpreter languages and scripting > External python modules > \texttt{python-flask}}
	\item biblioteka RPi.GPIO.
		\centeredmenu{Target packages > Interpreter languages and scripting > External python modules > \texttt{python-rpi-gpio}}
\end{itemize}

\subsection{Modyfikacja drzewa urządzeń}

Do powyższych opcji została dodatkowo dołączona również opcja
\centeredmenu{Host utilities > host dtc}
Jest ona używana do modyfikacji wygenerowanego przez Buildroot drzewa urządzeń tak, aby wspierane
były urządzenia audio. W tym celu najpierw binarny plik \texttt{.dtb} należy zdekompilować
poleceniem
\begin{lstlisting}[language=bash]
$ ${BR2_PATH}/output/build/host-dtc-1.4.1/dtc -I dtb -O dts
	-o bcm2708-rpi-b.dts bcm2708-rpi-b.dtb
\end{lstlisting}
Poniżej znajduje się treść łaty umożliwiającej na włączenie dźwięku.
\begin{lstlisting}[language=diff,caption=Łata włączająca wsparcie dla dźwięku]
@@ -489,7 +489,7 @@
 		audio {
 			compatible = "brcm,bcm2835-audio";
 			brcm,pwm-channels = <0x8>;
-			status = "disabled";
+			status = "okay";
 			pinctrl-names = "default";
 			pinctrl-0 = <0x18>;
 			phandle = <0x21>;
\end{lstlisting}
Po jej zastosowaniu obraz skompilowany został z powrotem poleceniem
\begin{lstlisting}[language=bash]
$ ${BR2_PATH}/output/build/host-dtc-1.4.1/dtc -I dts -O dtb
	-o bcm2708-rpi-b.dtb bcm2708-rpi-b.dts
\end{lstlisting}

\subsection{Realizacja szafy grającej}

Pliki aplikacji znajdują się w katalogu
\directory{overlay/var/www/jukebox}
w dołączonym archiwum.

Aplikacja została zrealizowana za pomocą języka skryptowego Python oraz mikroframeworka Flask
\cite{flask}, zawierającego wbudowany serwer HTTP i wsparcie dla szablonów HTML. Za odtwarzanie
mediów odpowiedzialna jest aplikacja VLC oraz dostarczone przez autorów bindingi do języka
Python \cite{vlc-bindings}. Strona używa także arkuszy stylów oraz skryptów JavaScript
zawartych we frameworku Materialize \cite{materialize}.

Właściwy kod serwera zawarty jest w pojedynczym pliku
\directory{overlay/var/www/jukebox/jukebox.py}.

Za kolejkowanie piosenek odpowiada klasa \texttt{JukeboxQueue}. Piosenki dodawane są do kolejki
tylko, jeśli jeszcze w niej się nie znajdują; w przeciwnym wypadku są przesuwane w górę listy.
W momencie zakończenia się jednej piosenki, następna brana jest ze szczytu kolejki.

Klasa \texttt{VlcJukebox} jest odpowiedzialna za odtwarzanie plików multimedialnych. W momencie
jej utworzenia tworzona jest instancja klasy \texttt{VlcInstance}, pozwalająca na odtwarzanie
multimediów.

\begin{lstlisting}[language=python,caption=Konstruktor klasy \texttt{VlcInstance}]
class VlcJukebox:
    def __init__(self, queue):
        self.instance = vlc.Instance()
        self.player = self.instance.media_player_new()
        event_manager = self.player.event_manager()
        event_manager.event_attach(vlc.EventType.MediaPlayerEndReached, 
		self.song_ended_callback)
        self.queue = queue
        self.queue_ended = True
        self.current = None
\end{lstlisting}

Odpowiednie metody klasy służą do kontroli odtwarzania:

\begin{itemize}
	\item Metoda \texttt{new\_song} odpowiada za odtworzenie kolejnej piosenki z kolejki.
	\item Metoda \texttt{queue\_updated} obsługuje zdarzenie dodania utworu do kolejki
		odtwarzania, wznawiając odtwarzanie, jeśli kolejka skończyła się wcześniej.
	\item Metoda \texttt{song\_ended\_callback} odtwarza kolejny utwór z kolejki
		po zakończeniu poprzedniego (jeśli taki istnieje).
	\item Metody \texttt{pause} i \texttt{skip} używane są do pauzowania oraz pomijania
		utworów.
	\item Metody \texttt{volume\_up} i \texttt{volume\_down} służą do kontroli głośności.
\end{itemize}

Funkcje zaadnotowane właściwością \texttt{@app.route} służą do obsługi poszczególnych
żądań HTTP. Jedynym widokiem aplikacji jest widok obsługiwany przez funkcję
\texttt{view\_tracks}, zawierający przeglądarkę plików. Do przeglądania poszczególnych
katalogów używana jest funkcja \texttt{os.walk}. Pliki są filtrowane tak, aby wyświetlane były
wyłącznie te z rozszerzeniem \texttt{.mp3}.

\begin{lstlisting}[language=python,caption=Metoda obsługująca przeglądarkę plików]
@app.route("/<path:folder>")
@app.route("/")
def view_tracks(folder=""):
    real_path = os.path.join(root_dir, folder)
    if not os.path.isdir(real_path):
        abort(404)
    root = None
    if (len(folder) > 0):
        root = os.path.abspath(os.path.join("/", folder, ".."))
    for (dirpath, dirnames, filenames) in os.walk(real_path):
        music_files = [filename for filename in filenames if is_mp3_file(filename)]
        return render_template("index.html", 
                folder=folder, 
                dirnames=dirnames, 
                filenames=music_files, 
                queue=[get_filename(path) for path in queue.get_contents()],
                root=root,
                current=get_filename(jukebox.current),
                icon=jukebox.play_button_icon())
\end{lstlisting}

Pozostałe metody odpowiadają za sterowanie odtwarzaczem z poziomu interfejsu WWW. Wszystkie
z nich wykonują odpowiednie czynności na instancji \texttt{VlcJukebox}, a następnie
przekierowują użytkownika na poprzednią stronę.

\begin{itemize}
	\item Dodanie pliku do kolejki (\texttt{play\_file}), pauza (\texttt{pause}),
		pominięcie utworu (\texttt{skip}) i kontrola głośności (\texttt{volume\_ctrl})
		są realizowane za pomocą zapytań GET.
	\item Metody \texttt{update\_queue} i \texttt{now\_playing} są używane do dynamicznej
		aktualizacji informacji wyświetlanych w interfejsie WWW. W źródle przeglądarki
		plików znajduje się kod JavaScript, który za pomocą jQuery okresowo wysyła żądania
		AJAX pod adresy \texttt{/now} i \texttt{/queue} i aktualizuje zawartość strony.
		\begin{lstlisting}[language=javascript, caption=Aktualizacja kolejki i obecnego utworu z pomocą jQuery]
function update() {
	$.ajax({
		url: "{{ url_for('.update_queue') }}",
		cache: false
	})
	.done(function(html) {
		$("#slide-out-contents").replaceWith(html)
	});
	$(".button-collapse").sideNav();

	$.ajax({
		url: "{{ url_for('.now_playing') }}",
		cache: false
	})
	.done(function(html) {
		$("#now-playing").replaceWith(html)
	});
}

setInterval(update, 5000)
		\end{lstlisting}
	\item Metody \texttt{upload} oraz \texttt{create\_dir} odpowiadają za zarządzanie utworami, tj.
		tworzenie katalogów i przesyłanie plików. Te czynności są wykonywane za pomocą żądań
		POST. Przy przesyłaniu plików dodatkowo wykorzystywana jest część FILES żądania.
\end{itemize}

Pozostałe trzy funkcje odpowiedzialne są za obsługę przycisków na dołączonej płytce. Ich obsługa
zrealizowana jest za pomocą biblioteki RPi.GPIO \cite{rpi-gpio}.

\begin{itemize}
	\item Pierwszy przycisk (pin 10) odpowiada za pauzowanie i wznawianie odtwarzania.
	\item Drugi przycisk (pin 22) odpowiada za pomijanie utworów.
	\item Trzeci przycisk (pin 27) służy jako modyfikator. Po przytrzymaniu go, przyciski 10 i 22
		służą do kontroli głośności odtwarzania. Ta funkcjonalność zrealizowana jest poprzez
		zarejestrowanie się na zdarzenie zmiany zbocza (w obie strony) i przełączanie
		globalnej flagi, oznaczającej aktywny tryb.
\end{itemize}

\begin{lstlisting}[language=python, caption=Rejestracja funkcji obsługujących wciśnięcia]
GPIO.setmode(GPIO.BCM)
GPIO.setup(10, GPIO.IN)
GPIO.add_event_detect(10, GPIO.FALLING, callback=pause_button_callback, bouncetime=200)
GPIO.setup(22, GPIO.IN)
GPIO.add_event_detect(22, GPIO.FALLING, callback=skip_button_callback, bouncetime=200)
GPIO.setup(27, GPIO.IN)
GPIO.add_event_detect(27, GPIO.BOTH, callback=volume_toggle_callback, bouncetime=200)

atexit.register(GPIO.cleanup)
\end{lstlisting}

Uruchomienie serwera odbywa się w głównej części skryptu, poprzez wywołanie funkcji
\begin{lstlisting}
app.run(host='0.0.0.0', port=80)
\end{lstlisting}

\begin{thebibliography}{9}

\bibitem{flask}
	Flask: A Python Microframework,
		\url{http://flask.pocoo.org/},
		BSD

\bibitem{materialize}
	Materialize CSS,
		\url{http://materializecss.com/},
		MIT

\bibitem{rpi-gpio}
	raspberry-gpio-python,
		SourceForge,
		\url{https://sourceforge.net/projects/raspberry-gpio-python/},
		MIT

\bibitem{vlc-bindings}
	Python bindings,
		VideoLAN Wiki,
		\url{https://wiki.videolan.org/python_bindings},
		GNU LGPL

\end{thebibliography}

\end{document}
