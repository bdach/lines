\documentclass[10pt,a4paper]{article}
\usepackage[T1]{fontenc}
\usepackage[utf8]{inputenc}
\usepackage{graphicx}
\usepackage{tabularx}
\usepackage{helvet}
\usepackage{bera}
\usepackage{hyperref}
\usepackage[a4paper,margin=1in]{geometry}
\usepackage[polish]{babel}
\usepackage{float}
\usepackage{listings}
\usepackage{menukeys}

\renewcommand\familydefault{\sfdefault}

\lstset{basicstyle=\ttfamily,
  showstringspaces=false,
  commentstyle=\color{gray},
  keywordstyle=\color{blue},
  stringstyle=\color{teal},
  captionpos=b
}

\title{Linux w systemach wbudowanych -- Laboratorium 3}
\author{Bartłomiej Dach}

\newcommand{\centeredmenu}[1]{
	\begin{center}
		\menu{#1}
	\end{center}
}

\begin{document}

\makeatletter
\begin{flushright}
	Warszawa, \@date
\end{flushright}
\begin{center}
	\LARGE{\@title}
\end{center}
\vspace{0.25cm}
Student: \@author
\makeatother

\section{Treść zadania}

Trzecie zadanie laboratoryjne składało się z następujących kroków:

\begin{enumerate}
	\item przygotowania systemu administracyjnego z systemem plików \verb+initramfs+,
		umożliwiającego:
		\begin{itemize}
			\item zarządzanie zawartością karty SD poprzez jej: partycjonowanie,
				formatowanie oraz naprawianie systemów plików znajdujących się
				na niej,
			\item kopiowanie i instalację obrazu systemu użytkowego z komputera
				laboratoryjnego.
		\end{itemize}
	\item podzielenie karty SD na następujące partycje:
		\begin{itemize}
			\item 1. partycja, z systemem plików VFAT, zawierająca obrazy systemów:
				ratunkowego, administracyjnego i użytkowego oraz skrypty
				uruchomieniowe bootloadera \verb+uboot+,
			\item 2. partycja, zawierająca system plików \verb+ext4+, będący głównym
				systemem plików dla systemu użytkowego,
			\item 3. partycja, na której będą umieszczane pliki dodane przez
				użytkowników i udostępniane poprzez system użytkowy.
		\end{itemize}
	\item przygotowanie systemu użytkowego, który zawiera serwer WWW (w moim przypadku
		zrealizowany we frameworku Django), sterowany poprzez interfejs WWW. Serwer ma
		umożliwiać pobieranie plików umieszczonych na trzeciej partycji karty SD oraz
		dodawanie nowych plików na tą partycję przez zalogowanych użytkowników,
	\item przygotowanie skryptu bootloadera, który umożliwi wybór uruchamianego systemu
		w zależności od stanu wciśniętych przycisków na dostarczonej płytce.
\end{enumerate}

\section{Procedura odtworzenia projektu z załączonego archiwum}

\subsection{Kompilacja systemu administracyjnego}

Pliki konfiguracyjne systemu administracyjnego znajdują się w katalogu \directory{files/admin}
w dostarczonym archiwum. W folderze tym znajduje się również skrypt powłoki, po którego
uruchomieniu komendą
\begin{lstlisting}[language=bash]
./make_admin.sh BR2PATH
\end{lstlisting}
gdzie \verb+BR2PATH+ oznacza katalog główny Buildroota, nastąpi przygotowanie obrazu systemu
administracyjnego.

\subsection{Kompilacja systemu użytkowego}

Pliki niezbędne do kompilacji systemu użytkowego znajdują się w folderze \directory{files/user}.
Poza łatą konfigurującą opcje obrazu (plik \texttt{user.patch}), folder ten zawiera paczkę
\texttt{uwsgi}, stanowiącą środowisko serwerowe aplikacji. Pliki konfiguracyjne paczki zostały
zaadoptowane z łaty \cite{uwsgi-package}.

W celu rozpoczęcia procesu kompilacji wystarczy wykonać dostarczony skrypt
\begin{lstlisting}[language=bash]
./make_user.sh BR2PATH
\end{lstlisting}
gdzie \verb+BR2PATH+ to katalog główny Buildroota. Wygenerowany zostanie obraz systemu, a także
skompresowana zawartość docelowa systemu plików dla systemu.

\subsection{Instalacja skryptu rozruchowego i obrazów}

Po kompilacji obu obrazów, należy skorzystać z systemu ratunkowego w celu umieszczenia obrazów
obu systemów na płytce.

\begin{itemize}
	\item Obraz systemu administracyjnego powinien mieć nazwę \texttt{zImage\_admin}.
	\item Obraz systemu użytkowego powinien mieć nazwę \texttt{zImage\_user}.
\end{itemize}

Skrypt uruchomieniowy pozwalający na wybór systemu za pomocą wejść GPIO znajduje się
w archiwum w pliku \directory{files/boot.script}. Skrypt ten należy skopiować na pierwszą
partycję karty SD, a następnie skompilować go poleceniem
\begin{lstlisting}
# mkimage -T script -C none -n "Start script" -d boot.script boot.scr
\end{lstlisting}

\subsection{Partycjonowanie dysku oraz wyodrębnienie użytkowego systemu plików z archiwum}

Po zainstalowaniu skryptów i skopiowaniu obrazów można uruchomić system administracyjny w celu
podziału karty SD na partycje. Do podziału można użyć dołączonego programu \texttt{parted}.
W wyniku podziału obok rozruchowej partycji VFAT muszą znaleźć się 2 partycje w systemie
\texttt{ext4}, przy czym druga z nich może mieć nie mniej niż ok. 150 MB.

Po podziale należy założyć na nowych partycjach systemy plików poleceniami
\begin{lstlisting}
# mkfs.ext4 /dev/mmcblk0p2
# mkfs.ext4 /dev/mmcblk0p3
\end{lstlisting}
Po wykonaniu powyższych poleceń można przystąpić do instalacji systemu plików na drugiej
partycji. Skompresowany lik z obrazem \texttt{rootfs.ext4.gz} można skopiować do pamięci
podręcznej płytki (tj. do \texttt{initramfs}) za pośrednictwem SSH. Po przesłaniu wypakowanie
na drugą partycję można dokonać poleceniem
\begin{lstlisting}
# gunzip rootfs.ext4.gz | dd of=/dev/mmcblk0p2 bs=2M
\end{lstlisting}
Po wykonaniu tych kroków system użytkowy powinien być gotowy do działania.

\section{Opis rozwiązania}

W przypadku obu systemów konfiguracja opiera się na podstawowej konfiguracji, która opisana
została w ,,Poradniku laboratoryjnym'':
\begin{itemize}
	\item port getty ustawiony na \texttt{ttyAMA0},
	\item wartość \emph{primary download site} wskazująca na lokalny serwer lustrzany,
	\item \emph{target ABI} na \texttt{EABI},
	\item użyty został zewnętrzny \emph{toolchain},
	\item w systemie administracyjnym dodany został \texttt{initramfs} wkompilowany
		w jądro,
	\item system plików został skompresowany za pomocą gzip.
\end{itemize}

\subsection{Konfiguracja systemu administracyjnego}

Nazwa hosta oraz wiadomość powitalna w systemie administracyjnym zostały zmienione, aby
informować użytkowników, na którym systemie pracują. Dodatkowo, w celu ułatwienia korzystania
z serwera SSH \texttt{dropbear} użytkownik \texttt{root} otrzymał hasło \texttt{LINSWlab3}.

Narzędzia do zarządzania partycjami rodziny FAT zostały dodane do systemu poprzez wybranie opcji
\centeredmenu{Target packages > Filesystem and flash utilities > \texttt{dosfstools}}
a następnie podopcji
\centeredmenu{Target packages > Filesystem and flash utilities > \texttt{dosfstools} > \texttt{fsck.fat}}
\centeredmenu{Target packages > Filesystem and flash utilities > \texttt{dosfstools} > \texttt{mkfs.fat}}
Analogiczne narzędzia dla systemów rodziny \emph{extended} zostały dodane poprzez zaznaczenie paczki
\centeredmenu{Target packages > Filesystem and flash utilities > \texttt{e2fsprogs}}
Po zaznaczeniu tej opcji zaznaczone zostało jeszcze narzędzie
\centeredmenu{Target packages > Filesystem and flash utilities > \texttt{e2fsprogs} > \texttt{resize2fs}}
w celu umożliwienia poprawienia ewentualnych błędów w partycjonowaniu dysków.

Wybranym narzędziem do partycjonowania dla obrazu administracyjnego jest paczka
\centeredmenu{Target packages > Hardware handling > \texttt{parted}}
zamiast w celu umożliwienia transferu plików przez SSH dodany został program
\centeredmenu{Target packages > Networking applications > \texttt{dropbear}}

\subsection{Konfiguracja systemu użytkowego}

W systemie użytkowym w celu uruchomienia aplikacji napisanej we frameworku Django dodano
interpreter Pythona:
\centeredmenu{Target packages > Interpreter languages and scripting > \texttt{python3}}
a następnie moduł zewnętrzny Django:
\centeredmenu{Target packages > Interpreter languages and scripting > External python modules > \texttt{python-django}}
W celu wdrożenia aplikacji do Buildroota została dodana paczka z serwerem aplikacyjnym
\texttt{uwsgi}, znajdująca się w menu
\centeredmenu{Target packages > Networking applications > \texttt{uwsgi}}
a następnie serwer WWW \texttt{nginx}:
\centeredmenu{Target packages > Networking applications > \texttt{nginx}}
razem z modułem współpracującym z \texttt{uwsgi}:
\centeredmenu{Target packages > Networking applications > \texttt{nginx} > http server > \texttt{ngx\_http\_uwsgi\_module}}

\subsection{Skrypt uruchomieniowy -- wybór systemu}

Skrypt pozwalający na wybór systemu do uruchomienia na podstawie przycisków znajduje się
w folderze \directory{files/boot.script} archiwum.

\begin{lstlisting}[caption=Zawartość skryptu uruchomieniowego]
gpio set 18
sleep 1
gpio clear 18
if gpio input 10 || gpio input 22 || gpio input 27; then
	gpio set 24
	fatload mmc 0:1 ${fdt_addr_r} bcm2708-rpi-b.dtb
	fatload mmc 0:1 ${kernel_addr_r} zImage_admin
	setenv bootargs "${bootargs_orig} console=ttyAMA0"
	bootz ${kernel_addr_r} - ${fdt_addr_r}
else
	gpio set 23
	fatload mmc 0:1 ${fdt_addr_r} bcm2708-rpi-b.dtb
	fatload mmc 0:1 ${kernel_addr_r} zImage_user
	setenv bootargs "${bootargs_orig} console=ttyAMA0 root=/dev/mmcblk0p2 rootwait"
	bootz ${kernel_addr_r} - ${fdt_addr_r}
fi
\end{lstlisting}
Na początku skryptu zapalana jest na 1 sekundę dioda w kolorze białym, oznaczająca, że będzie
wczytywany stan przycisków. Następnie następuje wczytanie stanu wszystkich przycisków; jeśli
wciśnięty jest dowolny z nich, uruchamiany jest serwer administracyjny, zaś w przeciwnym
przypadku -- system użytkownika. W obu przypadkach zapalane są diody sygnalizujące dokonany
wybór (odpowiednio: czerwona oraz zielona)

Oba systemy ładowane są z \emph{device tree} dla Raspberry Pi B. W obu przypadkach w opcjach
rozruchu ustawiana jest nazwa tty, który będzie używany jako konsola (w tym przypadku, getty,
z którego dane odczytywane są przez konwerter USB-UART).

W przypadku systemu użytkowego dodana jest opcja rozruchu, informująca system że główny
system plików znajduje się na drugiej partycji karty SD. Opcja \texttt{rootwait} dodatkowo
informuje jądro, że system plików może nie być gotowy (zamontowany) w momencie uruchamiania
systemu.

System użytkowy jest kompilowany z użyciem nakładki. Nakładka ta zawiera: pliki aplikacji
\texttt{filehost}, skrypty uruchomieniowe oraz pliki konfiguracyjne serwera WWW, a także
zmodyfikowany plik \directory{/etc/fstab}. Plik ten zawiera nową linię, która odpowiada
za automatyczne montowanie trzeciej partycji pod katalog \texttt{/files}:
\begin{lstlisting}[caption=Linia dodana do pliku \directory{/etc/fstab}]
# <file system>	<mount pt>	<type>	<options>	<dump>	<pass>
/dev/mmcblk0p3	/files		ext2	defaults	0	2
\end{lstlisting}

\subsection{Realizacja serwera w Django}

Aplikacja udostępniająca pliki znajduje się w katalogu
\directory{files/user-overlay/var/www/filehost}.
Składa się ona z głównego modułu \texttt{filehost} oraz modułu \texttt{files},
odpowiedzialnego za faktyczne udostępnianie plików. Jest ona dodawana do systemu docelowego
za pomocą systemu nakładek.

\subsubsection{Model w bazie danych}

Aplikacja nie udostępnia bezpośrednio plików z trzeciej partycji, lecz indeksuje dodane pliki
z pomocą bazy danych SQLite. Poniżej znajduje się model encji pojedynczego pliku w bazie.

\begin{lstlisting}[language=python,caption=Model encji \texttt{File}]
class File(models.Model):
    name = models.CharField(max_length=250)
    content = models.FileField()

    def __str__(self):
        return self.name
\end{lstlisting}

Plik reprezentowany jest za pomocą kolumny \texttt{name}, zawierającej krótki opis pliku (do 250
znaków) oraz \texttt{content}, zawierającej nazwę pliku na trzeciej partycji. Aplikacja jest
skonfigurowana tak, aby wszystkie dodane pliki znajdowały się na niej, za pomocą ustawienia
\begin{lstlisting}[language=bash,caption=Opcja w \directory{filehost/settings.py} ustawiająca lokalizację dodanych plików]
# Store files on the third partition
MEDIA_ROOT="/files/"
\end{lstlisting}

\subsubsection{Pobieranie plików}

Pliki pobierane są przez wysłanie zapytania GET na adres \texttt{http://192.168.x.x/download?id=[ID]},
gdzie \texttt{ID} stanowi identyfikator pliku w bazie. Pobieranie odbywa się poprzez otwarcie pliku,
wczytanie jego zawartości, owinięcie jej w obiekt \texttt{HttpResponse} oraz modyfikacja
nagłówka HTTP tak, aby przeglądarka zorientowała się, że powinna pobrać plik.
\begin{lstlisting}[language=python,caption=Widok w \directory{files/views.py} odpowiadający za pobieranie plików]
def download(request):
    requested_file = get_object_or_404(File, pk = request.GET.get('id'))
    requested_file.content.open()
    response = HttpResponse(requested_file.content.read())
    response['Content-Disposition'] = 'attachment; filename="{}"'
    	.format(requested_file.content.name)
    requested_file.content.close()
    return response
\end{lstlisting}

\subsubsection{Autoryzacja}

Do autoryzacji użytkowników wykorzystywane są wbudowane w Django mechanizmy i tabele \texttt{Users}
i \texttt{Groups}. Dostarczają one: obsługę logowania i rejestracji, a także sesyjność z użyciem
mechanizmu ciasteczek.

Za logowanie oraz rejestrację odpowiedzialne są widoki \texttt{LogInView} oraz
\texttt{RegisterView}. Korzystają one z wbudowanych w Django funkcji.

Dostępność do chronionych stron (np. strony odpowiadającej za wysyłanie plików) jest regulowana
poprzez wbudowany w Django komponent \texttt{LoginRequiredMixin}.

\subsubsection{Dodawanie plików}

Dodawanie plików jest osiągalne dla zalogowanych użytkowników poprzez wysłanie zapytania
na adres \texttt{http://192.168.x.x/upload}, gdzie w sekcji POST znajduje się żądany opis,
zaś w sekcji FILES -- zawartość dodawanego pliku.

\begin{lstlisting}[language=python,caption=Widok odpowiedzialny za dodawanie pliku]
class UploadView(LoginRequiredMixin, View):
    login_url = '/login'

    def post(self, request):
        form = UploadFileForm(request.POST, request.FILES)
        if form.is_valid():
            form.save()
            messages.success(request, 'File uploaded successfully.')
        else:
            messages.error(request, 'There was an error uploading the file.')
        return redirect('index')
\end{lstlisting}

\subsubsection{Interfejs użytkownika}

Interfejs został zrealizowany za pomocą wbudowanego w Django systemu szablonów oraz
formularzy. Pozwalają one na łatwe łączenie widoków z zawartością HTML. Używane szablony
znajdują się w katalogu \directory{files/templates/files}.

\begin{lstlisting}[language=html,caption=Przykład zastosowania szablonów i formularzy -- formularz dodawania plików]

<div class="panel panel-default panel-primary">
	<div class="panel-heading">Upload new file</div>
	<form action="" method="post" enctype="multipart/form-data">
		<div class="panel-body">
			
			{{ form.as_p }}
		</div>
		<div class="panel-footer">
			<input type="submit" class="btn btn-success" value="Upload" />
		</div>
	</form>
</div>

\end{lstlisting}

W powyższym przykładzie znajduje się formularz dodawania plików na głównej stronie. 
Owijająca fragment HTML klauzula \texttt{if} sprawdza, czy użytkownik zalogował się na stronę.
Atrybut \texttt{action} formularza zawiera wstawkę, która kieruje wynik formularza do widoku
\texttt{Upload}, co zapobiega wklejaniom w HTML łączy bezpośrednich. Django wspiera również
dodawanie tokenów zapobiegających przed atakami typu CSRF (ang. \emph{Cross-Site Request
Forgery}). Sam formularz renderowany jest automatycznie na podstawie modelu obiektu:

\begin{lstlisting}[language=python,caption=Przykładowy formularz -- dodawanie plików]
class UploadFileForm(forms.ModelForm):
    error_css_class = 'error'

    class Meta:
        model = File
        fields = ['name', 'content']
        widgets = {
                'name': forms.TextInput(attrs={'class': 'form-control'})
        }
\end{lstlisting}

Szablon wiązany jest z formularzem poprzez widok, który przekazuje do renderera stron tzw.
kontekst, zawierający wszystkie elementy, do których odnosi się w znacznikach \texttt{\{\{ \}\}}
szablon. Umożliwia to m.in. natychmiastową implementację paginacji (dzielenia listy plików
na wiele podstron, z ograniczoną ilością rekordw na każdej podstronie):
\begin{lstlisting}[language=python,caption=Widok łączący formularz oraz szablon]
class IndexView(View):
    def get(self, request):
        file_list = File.objects.all()
        paginator = Paginator(file_list, 25)

        page = request.GET.get('page')
        try:
            files = paginator.page(page)
        except PageNotAnInteger:
            files = paginator.page(1)
        except EmptyPage:
            files = paginator.page(paginator.num_pages)

        form = UploadFileForm()

        return render(request, 'files/index.html', 
		{
			"files": files,
			"form": form,
			"messages": messages.get_messages(request) 
		}
	)
\end{lstlisting}
Wygląd stron został ulepszony z pomocą frameworka Bootstrap \cite{bootstrap}, który dostarcza
gotowe arkusze stylów oraz komponenty w JavaScript.

\subsection{Konfiguracja środowiska uruchomieniowego serwera}

Aby serwer był uruchamiany z momencie startu systemu, dodany został \emph{init script}
uruchamiający uWSGI z odpowiednią konfiguracją:
\begin{lstlisting}[language=bash,caption=Zawartość pliku \directory{/etc/init.d/S49filehost}]
#!/bin/sh

case $1 in
	start)
		uwsgi --ini /var/www/filehost/uwsgi.ini
		;;
	stop)
		uwsgi --stop /tmp/filehost-master.pid
		;;
esac
\end{lstlisting}
Serwer uWSGI został skonfigurowany za pomocą następującego pliku \texttt{.ini}.
\begin{lstlisting}[caption=Zawartość pliku \directory{/var/www/filehost/uwsgi.ini}]
[uwsgi]

module=filehost.wsgi		 # nazwa modulu Pythona zawierajacego konfiguracje wsgi
master=true			 # czy instancja uWSGI jest instancja glowna
pidfile=/tmp/filehost-master.pid # lokalizacja pliku zawierajacego PIDy procesow serwera
socket=/var/www/filehost.sock    # plik gniazda, za posrednictwem ktorego odbywac sie
				 # bedzie polaczenie
processes=5			 # liczba procesow uwsgi
harakiri=20			 # maksymalne trwanie zadania w sekundach
max-requests=5000		 # maksymalna liczba zadan, po ktorych procesy powinny
				 # zostac przeladowane
vacuum=true			 # czyszczenie po zakonczeniu
daemonize=/var/log/filehost.log	 # uruchom jako daemon, wypisz log do pliku
buffer-size=32768		 # rozmiar bufora dla zadan
\end{lstlisting}

Aby połączyć plik gniazda stworzony przez uWSGI z serwerem WWW hostowanym przez \texttt{nginx},
dodany został do katalogu \directory{/etc/nginx/sites-enabled} plik \texttt{filehost}.
\begin{lstlisting}[caption=Zawartość pliku \directory{/etc/nginx/sites-enabled/filehost}]
# nginx configuration for the FileHost application

server {
	listen 0.0.0.0:80 default_server;
	listen [::]:80 default_server;

	root /var/www;

	server_name filehost;

	# Redirect requests to the root folder to the uWSGI socket file
	location / {
		include uwsgi_params;
		uwsgi_pass unix:/var/www/filehost.sock;
	}

	# Serve static files here
	location /static {
		autoindex on;
		alias /var/www/static;
	}
}
\end{lstlisting}
Według powyższej konfiguracji, wszystkie żądania HTTP do folderu innego niż \texttt{/static}
są przekierowywane do uWSGI. Pliki statyczne są serwowane wyłącznie przez \texttt{nginx},
co zapewnia większą wydajność.

\begin{thebibliography}{9}
	\bibitem{bootstrap}
		Bootstrap,
		\url{https://getbootstrap.com/},
		MIT

	\bibitem{uwsgi-package}
		Andrea Cappelli,
		\emph{[1/1] New package: uwsgi},
		Patchwork,
		\url{https://patchwork.ozlabs.org/patch/561902/}.
\end{thebibliography}

\end{document}
