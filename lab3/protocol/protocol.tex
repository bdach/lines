\documentclass[10pt,a4paper]{article}
\usepackage[T1]{fontenc}
\usepackage[utf8]{inputenc}
\usepackage{graphicx}
\usepackage{tabularx}
\usepackage{helvet}
\usepackage{bera}
\usepackage{hyperref}
\usepackage[a4paper,margin=1in]{geometry}
\usepackage[polish]{babel}
\usepackage{float}
\usepackage{listings}
\usepackage{menukeys}

\renewcommand\familydefault{\sfdefault}

\lstset{basicstyle=\ttfamily,
  showstringspaces=false,
  commentstyle=\color{gray},
  keywordstyle=\color{blue},
  stringstyle=\color{teal},
  captionpos=b
}

\title{Linux w systemach wbudowanych -- Laboratorium 3}
\author{Bartłomiej Dach}

\newcommand{\centeredmenu}[1]{
	\begin{center}
		\menu{#1}
	\end{center}
}

\begin{document}

\makeatletter
\begin{flushright}
	Warszawa, \@date
\end{flushright}
\begin{center}
	\LARGE{\@title}
\end{center}
\vspace{0.25cm}
Student: \@author
\makeatother

\section{Treść zadania}

Trzecie zadanie laboratoryjne składało się z następujących kroków:

\begin{enumerate}
	\item przygotowania systemu administracyjnego z systemem plików \verb+initramfs+,
		umożliwiającego:
		\begin{itemize}
			\item zarządzanie zawartością karty SD poprzez jej: partycjonowanie,
				formatowanie oraz naprawianie systemów plików znajdujących się
				na niej,
			\item kopiowanie i instalację obrazu systemu użytkowego z komputera
				laboratoryjnego.
		\end{itemize}
	\item podzielenie karty SD na następujące partycje:
		\begin{itemize}
			\item 1. partycja, z systemem plików VFAT, zawierająca obrazy systemów:
				ratunkowego, administracyjnego i użytkowego oraz skrypty
				uruchomieniowe bootloadera \verb+uboot+,
			\item 2. partycja, zawierająca system plików \verb+ext4+, będący głównym
				systemem plików dla systemu użytkowego,
			\item 3. partycja, na której będą umieszczane pliki dodane przez
				użytkowników i udostępniane poprzez system użytkowy.
		\end{itemize}
	\item przygotowanie systemu użytkowego, który zawiera serwer WWW (w moim przypadku
		zrealizowany we frameworku Django), sterowany poprzez interfejs WWW. Serwer ma
		umożliwiać pobieranie plików umieszczonych na trzeciej partycji karty SD oraz
		dodawanie nowych plików na tą partycję przez zalogowanych użytkowników,
	\item przygotowanie skryptu bootloadera, który umożliwi wybór uruchamianego systemu
		w zależności od stanu wciśniętych przycisków na dostarczonej płytce.
\end{enumerate}

\section{Procedura odtworzenia projektu z załączonego archiwum}

\subsection{Kompilacja systemu administracyjnego}

Pliki konfiguracyjne systemu administracyjnego znajdują się w katalogu \directory{files/admin}
w dostarczonym archiwum. W folderze tym znajduje się również skrypt powłoki, po którego
uruchomieniu komendą
\begin{lstlisting}[language=bash]
./make_admin.sh BR2PATH
\end{lstlisting}
gdzie \verb+BR2PATH+ oznacza katalog główny Buildroota, nastąpi przygotowanie obrazu systemu
administracyjnego.

\subsection{Kompilacja systemu użytkowego}

Pliki niezbędne do kompilacji systemu użytkowego znajdują się w folderze \directory{files/user}.
Poza łatą konfigurującą opcje obrazu (plik \texttt{user.patch}), folder ten zawiera paczkę
\texttt{uwsgi}, stanowiącą środowisko serwerowe aplikacji. Pliki konfiguracyjne paczki zostały
zaadoptowane z łaty \cite{uwsgi-package}.

W celu rozpoczęcia procesu kompilacji wystarczy wykonać dostarczony skrypt
\begin{lstlisting}[language=bash]
./make_user.sh BR2PATH
\end{lstlisting}
gdzie \verb+BR2PATH+ to katalog główny Buildroota. Wygenerowany zostanie obraz systemu, a także
skompresowana zawartość docelowa systemu plików dla systemu.

\subsection{Instalacja skryptu rozruchowego i obrazów}

Po kompilacji obu obrazów, należy skorzystać z systemu ratunkowego w celu umieszczenia obrazów
obu systemów na płytce.

\begin{itemize}
	\item Obraz systemu administracyjnego powinien mieć nazwę \texttt{zImage\_admin}.
	\item Obraz systemu użytkowego powinien mieć nazwę \texttt{zImage\_user}.
\end{itemize}

Skrypt uruchomieniowy pozwalający na wybór systemu za pomocą wejść GPIO znajduje się
w archiwum w pliku \directory{files/boot.script}. Skrypt ten należy skopiować na pierwszą
partycję karty SD, a następnie skompilować go poleceniem
\begin{lstlisting}
# mkimage -T script -C none -n "Start script" -d boot.script boot.scr
\end{lstlisting}

\subsection{Partycjonowanie dysku oraz wyodrębnienie użytkowego systemu plików z archiwum}

Po zainstalowaniu skryptów i skopiowaniu obrazów można uruchomić system administracyjny w celu
podziału karty SD na partycje. Do podziału można użyć dołączonego programu \texttt{parted}.
W wyniku podziału obok rozruchowej partycji VFAT muszą znaleźć się 2 partycje w systemie
\texttt{ext4}, przy czym druga z nich może mieć nie mniej niż ok. 150 MB.

Po podziale należy założyć na nowych partycjach systemy plików poleceniami
\begin{lstlisting}
# mkfs.ext4 /dev/mmcblk0p2
# mkfs.ext4 /dev/mmcblk0p3
\end{lstlisting}
Po wykonaniu powyższych poleceń można przystąpić do instalacji systemu plików na drugiej
partycji. Skompresowany lik z obrazem \texttt{rootfs.ext4.gz} można skopiować do pamięci
podręcznej płytki (tj. do \texttt{initramfs}) za pośrednictwem SSH. Po przesłaniu wypakowanie
na drugą partycję można dokonać poleceniem
\begin{lstlisting}
# gunzip rootfs.ext4.gz | dd of=/dev/mmcblk0p2 bs=2M
\end{lstlisting}
Po wykonaniu tych kroków system użytkowy powinien być gotowy do działania.

\section{Opis rozwiązania}

W przypadku obu systemów konfiguracja opiera się na podstawowej konfiguracji, która opisana
została w ,,Poradniku laboratoryjnym'':
\begin{itemize}
	\item port getty ustawiony na \texttt{ttyAMA0},
	\item wartość \emph{primary download site} wskazująca na lokalny serwer lustrzany,
	\item \emph{target ABI} na \texttt{EABI},
	\item użyty został zewnętrzny \emph{toolchain},
	\item w systemie administracyjnym dodany został \texttt{initramfs} wkompilowany
		w jądro,
	\item system plików został skompresowany za pomocą gzip.
\end{itemize}

\subsection{Konfiguracja systemu administracyjnego}

Nazwa hosta oraz wiadomość powitalna w systemie administracyjnym zostały zmienione, aby
informować użytkowników, na którym systemie pracują. Dodatkowo, w celu ułatwienia korzystania
z serwera SSH \texttt{dropbear} użytkownik \texttt{root} otrzymał hasło \texttt{LINSWlab3}.

\subsection{Konfiguracja systemu użytkowego}

\subsection{Skrypt uruchomieniowy -- wybór systemu}

\subsection{Realizacja serwera w Django}

\subsection{Konfiguracja środowiska uruchomieniowego serwera}

\begin{thebibliography}{9}
	\bibitem{uwsgi-package}
		Andrea Cappelli,
		\emph{[1/1] New package: uwsgi},
		Patchwork,
		\url{https://patchwork.ozlabs.org/patch/561902/}.
\end{thebibliography}

\end{document}
